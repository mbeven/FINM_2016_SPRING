%%%%%%%%%%%%%%%%%%%%%%%%%%%%%%%%%%%%%%%%%
% Short Sectioned Assignment
% LaTeX Template
% Version 1.0 (5/5/12)
%
% This template has been downloaded from:
% http://www.LaTeXTemplates.com
%
% Original author:
% Frits Wenneker (http://www.howtotex.com)
%
% License:
% CC BY-NC-SA 3.0 (http://creativecommons.org/licenses/by-nc-sa/3.0/)
%
%%%%%%%%%%%%%%%%%%%%%%%%%%%%%%%%%%%%%%%%%

%----------------------------------------------------------------------------------------
%	PACKAGES AND OTHER DOCUMENT CONFIGURATIONS
%----------------------------------------------------------------------------------------

\documentclass[paper=a4, fontsize=11pt]{scrartcl} % A4 paper and 11pt font size

\usepackage[T1]{fontenc} % Use 8-bit encoding that has 256 glyphs
\usepackage{fourier} % Use the Adobe Utopia font for the document - comment this line to return to the LaTeX default
\usepackage[english]{babel} % English language/hyphenation
\usepackage{amsmath,amsfonts,amsthm,mathtools,amssymb} % Math packages
\usepackage{dsfont} %double stroke
\usepackage[margin=1.75cm]{geometry}
\usepackage{multicol}
\usepackage{setspace}
\usepackage{graphicx}
\usepackage{setspace}
\onehalfspacing
\usepackage{multicol}
\allowdisplaybreaks
\usepackage{hyperref}

% R code
\usepackage{listings}

    \lstset{
    language=R,
    basicstyle=\scriptsize\ttfamily,
    commentstyle=\ttfamily\color{gray},
    numbers=left,
    numberstyle=\ttfamily\color{gray}\footnotesize,
    stepnumber=1,
    numbersep=5pt,
    backgroundcolor=\color{white},
    showspaces=false,
    showstringspaces=false,
    showtabs=false,
    frame=single,
    tabsize=2,
    captionpos=b,
    breaklines=true,
    breakatwhitespace=false,
    title=\lstname,
    escapeinside={},
    keywordstyle={},
    morekeywords={}
    }
% Binomial tree
\usepackage{tikz}
\usetikzlibrary{matrix}

% Permutations and combinations 
\newcommand*{\Perm}[2]{{}^{#1}\!P_{#2}}%
\newcommand*{\Comb}[2]{{}^{#1}C_{#2}}%

\usepackage{sectsty} % Allows customizing section commands
\allsectionsfont{\raggedright \normalfont} % Make all sections left, the default font and small caps
\renewcommand{\thesubsection}{\alph{subsection}} % Make the subsections letters
\newcommand{\rreduce}[2]{\mathop{\longrightarrow}_{\tiny{#1}}}

\usepackage{float} % picture placement
\usepackage{subfig} % picture placement
\usepackage{fancyhdr} % Custom headers and footers
\pagestyle{fancyplain} % Makes all pages in the document conform to the custom headers and footers
\fancyhead{} % No page header - if you want one, create it in the same way as the footers below
\fancyfoot[L]{} % Empty left footer
\fancyfoot[C]{} % Empty center footer
\fancyfoot[R]{\thepage} % Page numbering for right footer
\renewcommand{\headrulewidth}{0pt} % Remove header underlines
\renewcommand{\footrulewidth}{0pt} % Remove footer underlines
\setlength{\headheight}{13.6pt} % Customize the height of the header

\numberwithin{equation}{section} % Number equations within sections (i.e. 1.1, 1.2, 2.1, 2.2 instead of 1, 2, 3, 4)
\numberwithin{figure}{section} % Number figures within sections (i.e. 1.1, 1.2, 2.1, 2.2 instead of 1, 2, 3, 4)
\numberwithin{table}{section} % Number tables within sections (i.e. 1.1, 1.2, 2.1, 2.2 instead of 1, 2, 3, 4)

\setlength\parindent{0pt} % Removes all indentation from paragraphs - comment this line for an assignment with lots of text

%----------------------------------------------------------------------------------------
%	TITLE SECTION
%----------------------------------------------------------------------------------------

\newcommand{\horrule}[1]{\rule{\linewidth}{#1}} % Create horizontal rule command with 1 argument of height

\title{	
\normalfont \normalsize 
\textsc{University of Chicago | Financial Mathematics} \\ [25pt] % Your university, school and/or department name(s)
\horrule{0.5pt} \\[0.4cm] % Thin top horizontal rule
\huge FINM 37300: Homework 3 \\ % The assignment title
\horrule{2pt} \\[0.5cm] % Thick bottom horizontal rule
}

\author{Michael Beven - 455613} % Your name

\date{\normalsize\today} % Today's date or a custom date

\begin{document}

\maketitle % Print the title

%----------------------------------------------------------------------------------------
%	PROBLEM 16
%----------------------------------------------------------------------------------------

\section*{16.}

We have CLP \textit{onshore} deposit rates.  We cannot decide whether the spot is greater or less than the forward, since the USDCLP is non-deliverable.  

%----------------------------------------------------------------------------------------
%	PROBLEM 17
%----------------------------------------------------------------------------------------

\section*{17.}

We don't cross the bid ask spread twice, therefore use the 3M and 6M bid rates to buy/sell:

3M bid:

\begin{align*}
8.1510 \frac{1-0.70\% \frac{92}{360}}{1+0.65\% \frac{92}{360}} = 8.12331
\end{align*}

6M bid:

\begin{align*}
8.1510 \frac{1-0.44\% \frac{92}{360}}{1+0.93\% \frac{92}{360}} = 8.094809
\end{align*}

Therefore the swap points are:

\begin{align*}
(8.12331 - 8.094809)\times 10,000 = 281.17 \text{swap points}
\end{align*}

%----------------------------------------------------------------------------------------
%	PROBLEM 18
%----------------------------------------------------------------------------------------

\section*{18.}

Since we don't cross the bid ask spread twice, just take the difference in the forward points on the bid side.  Therefore, 13-3 = 10 swap points.  

%----------------------------------------------------------------------------------------
%	PROBLEM 19
%----------------------------------------------------------------------------------------

\section*{19.}

I would quote the 12 month rate, because otherwise there would be an arbitrage profit by buying at any lower rate I quote and taking the short side with the 12 month rate.   

%----------------------------------------------------------------------------------------
%	PROBLEM 20
%----------------------------------------------------------------------------------------

\section*{20.}

The market suggests the Canadian interest rate is going down relative to the US interest rate, therefore we wouldn't participate in a quote.  Also, the spread between the bid and ask is very narrow compared to question 19, so there is more uncertainty in where the exchange rate will go. 


%----------------------------------------------------------------------------------------

\end{document}
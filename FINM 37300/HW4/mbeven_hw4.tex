%%%%%%%%%%%%%%%%%%%%%%%%%%%%%%%%%%%%%%%%%
% Short Sectioned Assignment
% LaTeX Template
% Version 1.0 (5/5/12)
%
% This template has been downloaded from:
% http://www.LaTeXTemplates.com
%
% Original author:
% Frits Wenneker (http://www.howtotex.com)
%
% License:
% CC BY-NC-SA 3.0 (http://creativecommons.org/licenses/by-nc-sa/3.0/)
%
%%%%%%%%%%%%%%%%%%%%%%%%%%%%%%%%%%%%%%%%%

%----------------------------------------------------------------------------------------
%	PACKAGES AND OTHER DOCUMENT CONFIGURATIONS
%----------------------------------------------------------------------------------------

\documentclass[paper=a4, fontsize=11pt]{scrartcl} % A4 paper and 11pt font size

\usepackage[T1]{fontenc} % Use 8-bit encoding that has 256 glyphs
\usepackage{fourier} % Use the Adobe Utopia font for the document - comment this line to return to the LaTeX default
\usepackage[english]{babel} % English language/hyphenation
\usepackage{amsmath,amsfonts,amsthm,mathtools,amssymb} % Math packages
\usepackage{dsfont} %double stroke
\usepackage[margin=1.75cm]{geometry}
\usepackage{multicol}
\usepackage{setspace}
\usepackage{graphicx}
\usepackage{setspace}
\onehalfspacing
\usepackage{multicol}
\allowdisplaybreaks
\usepackage{hyperref}

% R code
\usepackage{listings}

    \lstset{
    language=R,
    basicstyle=\scriptsize\ttfamily,
    commentstyle=\ttfamily\color{gray},
    numbers=left,
    numberstyle=\ttfamily\color{gray}\footnotesize,
    stepnumber=1,
    numbersep=5pt,
    backgroundcolor=\color{white},
    showspaces=false,
    showstringspaces=false,
    showtabs=false,
    frame=single,
    tabsize=2,
    captionpos=b,
    breaklines=true,
    breakatwhitespace=false,
    title=\lstname,
    escapeinside={},
    keywordstyle={},
    morekeywords={}
    }
% Binomial tree
\usepackage{tikz}
\usetikzlibrary{matrix}

% Permutations and combinations 
\newcommand*{\Perm}[2]{{}^{#1}\!P_{#2}}%
\newcommand*{\Comb}[2]{{}^{#1}C_{#2}}%

\usepackage{sectsty} % Allows customizing section commands
\allsectionsfont{\raggedright \normalfont} % Make all sections left, the default font and small caps
\renewcommand{\thesubsection}{\alph{subsection}} % Make the subsections letters
\newcommand{\rreduce}[2]{\mathop{\longrightarrow}_{\tiny{#1}}}

\usepackage{float} % picture placement
\usepackage{subfig} % picture placement
\usepackage{fancyhdr} % Custom headers and footers
\pagestyle{fancyplain} % Makes all pages in the document conform to the custom headers and footers
\fancyhead{} % No page header - if you want one, create it in the same way as the footers below
\fancyfoot[L]{} % Empty left footer
\fancyfoot[C]{} % Empty center footer
\fancyfoot[R]{\thepage} % Page numbering for right footer
\renewcommand{\headrulewidth}{0pt} % Remove header underlines
\renewcommand{\footrulewidth}{0pt} % Remove footer underlines
\setlength{\headheight}{13.6pt} % Customize the height of the header

\numberwithin{equation}{section} % Number equations within sections (i.e. 1.1, 1.2, 2.1, 2.2 instead of 1, 2, 3, 4)
\numberwithin{figure}{section} % Number figures within sections (i.e. 1.1, 1.2, 2.1, 2.2 instead of 1, 2, 3, 4)
\numberwithin{table}{section} % Number tables within sections (i.e. 1.1, 1.2, 2.1, 2.2 instead of 1, 2, 3, 4)

\setlength\parindent{0pt} % Removes all indentation from paragraphs - comment this line for an assignment with lots of text

%----------------------------------------------------------------------------------------
%	TITLE SECTION
%----------------------------------------------------------------------------------------

\newcommand{\horrule}[1]{\rule{\linewidth}{#1}} % Create horizontal rule command with 1 argument of height

\title{	
\normalfont \normalsize 
\textsc{University of Chicago | Financial Mathematics} \\ [25pt] % Your university, school and/or department name(s)
\horrule{0.5pt} \\[0.4cm] % Thin top horizontal rule
\huge FINM 37300: Homework 4 \\ % The assignment title
\horrule{2pt} \\[0.5cm] % Thick bottom horizontal rule
}

\author{Michael Beven - 455613} % Your name

\date{\normalsize\today} % Today's date or a custom date

\begin{document}

\maketitle % Print the title

%----------------------------------------------------------------------------------------
%	PROBLEM 21
%----------------------------------------------------------------------------------------

\section*{21.}

I have written my solution in R.  All comments are commented in the code:

\begin{verbatim}[commandchars=\\\{\}]
# set parameters for Garman Kolhagen formula
days = as.numeric(difftime(strptime("11.10.2016", format = "%d.%m.%Y"),
               strptime("11.04.2016", format = "%d.%m.%Y"),units="days")) # trade date to expiry date
tau = days/365
r_AUD = 0.0215 # aud deposit rate
r_USD = 0.0038 # usd deposit rate
AUD_ACT = 365 # day count convention
USD_ACT = 360 # day count convention
sig = 0.128 # implied volatility
K = 0.7400 # strike
S = 0.7540 # spot
notional = 5*10^7 # notional amount

# implement GK formula
Pd = 1/(1+r_USD*days/USD_ACT) # present value of usd
Fwd = S*((1+r_USD*days/USD_ACT)/(1+r_AUD*days/AUD_ACT)) # forward
d1 = (log(Fwd/K) + 0.5*sig^2*tau)/(sig*sqrt(tau))
d2 = (log(Fwd/K) - 0.5*sig^2*tau)/(sig*sqrt(tau))
w = -1 # call or put toggle
p = Pd*w*(Fwd*pnorm(w*d1)-K*pnorm(w*d2)) # price 
pnumccy = p

# USD premium
cat('USD premium: ', notional*pnumccy)

******************************
USD premium:  1164897
******************************

# USD pips
cat('USD pips:', 1*10^4*pnumccy)

******************************
USD pips: 232.9794
******************************

# USD %
cat('USD %:', 100*pnumccy/K)

******************************
USD %: 3.14837
******************************

# AUD premium
cat('AUD premium: ', notional*pnumccy/S)

******************************
AUD premium:  1544956
******************************

# AUD pips
cat('AUD pips:', 1*10^4*pnumccy/K/S)

******************************
AUD pips: 417.5557
******************************

# AUD %
cat('AUD %:', 100*pnumccy/S)

******************************
AUD %: 3.089912
******************************

\end{verbatim}

%----------------------------------------------------------------------------------------
%	PROBLEM 22
%----------------------------------------------------------------------------------------

\section*{22.}

The call delta is $N(d_1)$ and the put delta is $-N(-d_1)$.  We also know that $N(x) + N(-x) = 1$.  Hence:

\begin{align*}
\text{call + put} = N(d_1) - N(-d_1) = 2N(d_1)-1
\end{align*}

If we set this to 0,

\begin{align*}
2N(d_1) &= 1\\
N(d_1) &= \frac{1}{2}\\
d_1 &= 0
\end{align*}

This implies that:

\begin{align*}
\text{log}(F/K) + \frac{1}{2}\sigma^2 \tau &= 0\\
\therefore F/K &= e^{-\frac{\sigma^2 \tau}{2}}\\
K &= Fe^{-\frac{\sigma^2 \tau}{2}}
\end{align*}

Therefore, plugging in our parameters we get: $K \approx 0.7505$.  

%----------------------------------------------------------------------------------------
%	PROBLEM 23
%----------------------------------------------------------------------------------------

\section*{23.}

My answer is \textbf{b)}. For a call, 

\begin{align*}
\text{abs}(N(d_1)) = N(d_1)
\end{align*}

For a put,

\begin{align*}
\text{abs}(-N(-d_1)) =  \text{abs}(N(d_1)-1) = 1 - N(d_1)
\end{align*}

$\sigma$ and $\tau$ do not vary.  Therefore the delta is determined by $log(F/K)$.  Hence, the absolute value of the call is directly positively proportional to $log(F/K)$ and the absolute value of the put is directly negatively proportional to $log(F/K)$.  \\

Therefore, in terms of absolute values of the calls and puts, we want to find the option that is deepest in the money.  a) is at the money and c) is out of the the money.  b) and d) are in the money, however b) is deeper in the money.  
%----------------------------------------------------------------------------------------
%	PROBLEM 24
%----------------------------------------------------------------------------------------

\section*{24.}

My answer is \textbf{b)}.  The option is deep in-the-money and expiry is very close, hence the option value curve would have approached the payoff curve. So its delta is approximately equal to 1.  Therefore the delta hedge would be to sell US\$100million.

%----------------------------------------------------------------------------------------
%	PROBLEM 25
%----------------------------------------------------------------------------------------

\section*{25.}

My answer is \textbf{a)}.  The option sensitivity increases as time to expiration increases.  Also, the closer the option is to the strike, the higher the chance of it flipping between in-the-money and out-the-money.  

%----------------------------------------------------------------------------------------

\end{document}